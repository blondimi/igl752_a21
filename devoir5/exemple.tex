\documentclass{article}

%% Packages
\usepackage[utf8]{inputenc}
\usepackage[french]{babel}
\usepackage{amsmath,amssymb}
\usepackage{enumerate}
\usepackage{bm}
\usepackage{tikz}
\usetikzlibrary{automata, petri, positioning}

% LTL/CTL
\newcommand{\true}{{\mathit{vrai}}}
\newcommand{\false}{{\mathit{faux}}}
\newcommand{\X}{{\mathsf{X}}}
\newcommand{\F}{{\mathsf{F}}}
\newcommand{\G}{{\mathsf{G}}}
\newcommand{\U}{\mathrel{\mathsf{U}}}

%% Réseaux de Petri
\newcommand{\pn}{\mathcal{N}}
\renewcommand{\vec}[1]{\bm{#1}}
\newcommand{\mat}[1]{\mathbf{#1}}
\newcommand{\Post}{\mathrm{Post}}

%% Chaînes de Markov
\newcommand{\init}{\vec{\mathrm{init}}}
\newcommand{\Pmat}{\mat{P}}
\newcommand{\M}{\mathcal{M}}
\newcommand{\cyl}[1]{\mathrm{cyl}(#1)}
\newcommand{\Prob}{\mathbb{P}}
\renewcommand{\P}{\mathcal{P}}

%% Titre
\title{IGL502/752: devoir 5}
\author{Foo McBar}
\date{7 décembre 2021}

\begin{document}
\maketitle

%% Contenu
\section*{Question 1}

\begin{center}
  \begin{tikzpicture}[auto, node distance=2cm, very thick]
    \node[place,      label=above:$p$, tokens=2]   (p) {};
    \node[transition, label=above:$t$, right of=p] (t) {};
    \node[place,      label=above:$q$, right of=t] (q) {};

    \path[->]
    (p) edge[bend left] node {} (t)
    (t) edge[bend left] node {} (p)
    (t) edge[]          node {} (q)
    ;
  \end{tikzpicture}
\end{center}

\section*{Question 2}

\begin{enumerate}[(a)]
  
\item ...

  \begin{center}
    \begin{tikzpicture}[auto, node distance=1cm, very thick, initial text=]
      \node[] (11) {$(1, 1)$};
      \node[below  left=0.5cm and 0.15cm of 11] (03) {$(0, 3)$};
      \node[below right=0.5cm and 0.15cm of 11] (20) {$(2, 0)$};
      \node[below right=0.5cm and 0.15cm of 03] (1w) {$(1, \omega)$};
      \node[below  left=0.5cm and 0.15cm of 1w] (0w) {$(0, \omega)$};
      \node[below right=0.5cm and 0.15cm of 1w] (ww) {$(\omega, \omega)$};

      \path[->]
      (11) edge node[swap] {$s$} (03)
      (11) edge node       {$t$} (20)
      (03) edge node       {$t$} (1w)
      (1w) edge node[swap] {$s$} (0w)
      (1w) edge node       {$t$} (ww)
      (0w) edge node[swap] {$t$} (ww)
      (ww) edge[loop right] node[] {$s, t$} ()
      ;
    \end{tikzpicture}
  \end{center}
  
\item $\Post^*(\vec{m})$...
  
\item $\vec{m}_0$, $\vec{m}_1$, $\vec{m}_2$, ...
  
\item ...
  
\end{enumerate}

\section*{Question 3}

\begin{enumerate}[(a)]
\item ...

  \begin{center}
    \begin{tabular}{c|l|ll}
      \textbf{Itér.} & \textbf{Base $B$}
      & \multicolumn{2}{c}{\textbf{Prédécesseurs}} \\
      
      \hline
      
      0 & $\{(0, 2)\}$ & $(0, 2)_{t_1} = (4, 4)$ & $(0, 2)_{t_2} = (2,
      1)$ \\

      \hline

      1 & $\{(0, 2), (2, 1)\}$ & $(2, 1)_{t_1} = (4, 3)$ & $(2,
      1)_{t_2} = (3, 0)$ \\

      \hline
      
      2 & $\{(0, 2), (2, 1), (3, 0)\}$ & $(3, 0)_{t_1} = (4, 2)$ &
      $(3, 0)_{t_2} = (4, 0)$ \\

      \hline
      
      3 & $\{(0, 2), (2, 1), (3, 0)\}$ & \multicolumn{2}{c}{base
        inchangée}
    \end{tabular}
  \end{center}

\item ...
\end{enumerate}

\section*{Question 4}

  \begin{center}
    \begin{tikzpicture}[auto, node distance=1.25cm, very thick,
                        transform shape, scale=0.85]

      \node[state, initial, initial text=$1/2$] (s0) {$s_0$};
      \node[state, right=3cm of s0]             (s1) {$s_1$};
      \node[state, right=3cm of s1]             (s2) {$s_2$};
      \node[state, right=3cm of s2]             (s3) {$s_3$};

      \node[state, below=3.25cm of s0]           (s4) {$s_4$};
      \node[state, right=3cm of s4, initial below,
            initial text=$1/2$] (s5) {$s_5$};
      \node[state, right=3cm of s5]             (s6) {$s_6$};
      \node[state, right=3cm of s6]             (s7) {$s_7$};

      \node[above=0pt of s0] {$\{p, q\}$};
      \node[above=0pt of s1] {$\{q\}$};
      \node[above=0pt of s2] {$\{p\}$};
      \node[above=0pt of s3] {$\{p\}$};

      \node[below=0pt of s4] {$\emptyset$};
      \node[below=0pt of s5, xshift=-15pt] {$\{p\}$};
      \node[below=0pt of s6] {$\{p\}$};
      \node[below=0pt of s7] {$\{q\}$};
      
      \path[->]
      (s0)  edge[]           node {$1$}   (s1)
      (s1)  edge[loop right] node[swap] {$3/4$} ()
      (s1)  edge[]           node {$1/4$} (s4)
      (s2)  edge[]           node {$3/4$} (s3)
      (s2)  edge[bend right=15] node[swap] {$1/8$} (s5)
      (s2)  edge[]           node {$1/8$} (s7)
      (s3)  edge[loop right] node[] {$1$}  ()

      (s4)  edge[] node[] {$1$} (s0)
      (s5)  edge[bend right=15] node[swap] {$1/2$} (s2)
      (s5)  edge[] node[] {$1/4$} (s4)
      (s5)  edge[] node[swap]       {$1/4$} (s6)
      (s6)  edge[loop above] node[] {$1/2$} (s6)
      (s6)  edge[bend right=20] node[swap] {$1/2$} (s7)
      (s7)  edge[bend right=20] node[swap] {$1$}   (s6)
      ;
    \end{tikzpicture}
  \end{center}
  \begin{enumerate}[(a)]
    
  \item Dites si la probabilité $\Prob(\F \G\, p)$ est $>0$, $=0$,
    $=1$ et/ou $< 1$. Justifiez.

  \item Dites si la probabilité $\Prob(\G \F\, p)$ est $>0$, $=0$,
    $=1$ et/ou $< 1$. Justifiez.

  \item Dites si tous les états initiaux de $\M$ satisfont la formule
    PCTL $\P_{\geq 3/4}(p \U \P_{\geq 1/2}(\X\, q))$. Justifiez.
\end{enumerate}

\end{document}
